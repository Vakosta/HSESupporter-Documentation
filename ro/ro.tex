\documentclass{../includes/TechDoc}
\usepackage[T1]{fontenc}
\usepackage[utf8]{inputenc}
\usepackage[pdftex]{graphicx}
\DeclareGraphicsRule{*}{mps}{*}{}

\title{Мобильное приложение для сообщений о бытовых проблемах студентов в общежитии}
\author{Студент группы БПИ-194}{В. А. Анненков}
\academicTeacher{Доцент департамента программной инженерии факультета компьютерных наук}{Х. М. Салех}

\documentTitle{Руководство оператора}
\documentCode{RU.17701729.02.07-01 ТЗ 01--1}

\begin{document}
    \maketitle

    \begin{abstract}
        Руководство оператора — это документ, назначение которого — предоставить людям помощь в использовании некоторого программного продукта.

        Настоящее Руководство оператора предназначено для правильной организации работы с приложением «Мобильное приложение для сообщений о бытовых проблемах студентов в общежитии».

        Данное руководство оператора содержит следующие разделы: «Назначение программы», «Условия выполнения программы», «Выполнение программы», «Сообщения оператору» и «Приложения».

        В разделе «Назначение программы» указаны сведения о назначении программы и информация о функциях и принципе эксплуатации программы.

        В разделе «Условия выполнения программы» содержится информацию об условиях, необходимых для выполнения данной программы (минимальный состав аппаратурных и программных средств).

        В разделе «Выполнение программы» указана последовательность действий оператора, обеспечивающих загрузку, запуск, выполнение и завершение программы, описание интерфейса, с помощью которого оператор осуществляет загрузку и управляет выполнением программы, а также реакция программы на эти команды.

        В разделе «Сообщения оператору» указаны тексты сообщений, выдаваемые в ходе выполнения программы, описание их содержания и соответствующие действия оператора.
    \end{abstract}

    \newpage

    \tableofcontents


    \section{Назначение программы}

    \subsection{Функциональное назначение}

    Функциональным назначением программы является помощь и консультирование студентов по вопросам, связанным
    с проживанием в общежитии. Студент может задать вопрос агенту поддержки, пообщаться в общем чате с другими
    студентами из этого же общежития, узнать последние новости общежития. Агент поддержки имеет доступ к панели
    администрирования, через которую он может создавать объявления, модерировать пользователей и их действия. Отвечать
    на обращения пользователей агент поддержки может с помощью мобильного приложения.

    \subsection{Эксплуатационное назначение}

    Программа может быть использована высшим учебным заведением для обеспечения централизованной связи со студентами, а
    также для информирования студентов о последних новостях в своих общежитиях.

    \subsection{Состав выполняемых функций}

    Программа должна обеспечивать возможность выполнения следующих функций:\\

    Для клиента:
    \begin{itemize}[noitemsep]
        \item регистрация и авторизация;
        \item просмотр страницы с новостями;
        \item просмотр и редактирование собственного профиля;
        \item просмотр/создание/удаление/дополнение своих обращений;
        \item просмотр и взаимодействие с общим чатом.
    \end{itemize}

    Для агента поддержки:
    \begin{itemize}[noitemsep]
        \item регистрация и авторизация в приложении;
        \item регистрация и авторизация в панели администрирования;
        \item просмотр и изменение списка новостей;
        \item просмотр/изменение/удаление/дополнение любых обращений.
    \end{itemize}

    Для администратора:
    \begin{itemize}[noitemsep]
        \item всё, что доступно агенту поддержки;
        \item управление существующими агентами поддержки;
        \item возможность назначить или отозвать у пользователя статус агента поддержки;
        \item полный доступ без ограничений к панели администрирования.
    \end{itemize}


    \section{Условия выполнения программы}

    \subsection{Минимальный состав аппаратных средств}

    Для исправной работы программы требуются следующие характеристики:
    \begin{itemize}[noitemsep]
        \item мобильный телефон с операционной системой iOS;
        \item не менее 9 МБ оперативной памяти;
        \item интернет-соединение.
    \end{itemize}

    \subsection{Минимальный состав программных средств}

    Для работы программы необходим следующий состав программных средств:
    \begin{itemize}[noitemsep]
        \item операционная система iOS 11 и выше.
    \end{itemize}

    \subsection{Требования к пользователю}

    Для работы с данной программой от конечного пользователя требуются базовые навыки владения
    операционной системой iOS.


    \section{Выполнение программы}

    \subsection{Запуск программы}

    \ldots

    \subsection{Описание интерфейса программы}

    \subsubsection{Авторизация}

    \begin{enumerate}
        \item После первого запуска программы пользователь видит экран авторизации (Рис. 1).
        \begin{figure}[h]
            \centering
            \frame{\includegraphics[width=0.4\linewidth]{images/login.png}}
            \caption{Экран авторизации}
            \label{fig:login}
        \end{figure}

        \newpage
        \item Для авторизации требуется сделать следующие действия:
        \begin{enumerate}
            \item Ввести Email в домене @hse.ru или @edu.hse.ru и нажать кнопку <<Войти>> (Рис.~\ref{ris:login_code});
            \item Ввести код подтверждения, который был отправлен на указанную почту и нажать кнопку <<Войти>>;
            \item Указать действительные персональные данные (Рис.~\ref{ris:personal_data}).
        \end{enumerate}
        \begin{figure}[h]
            \begin{center}
                \begin{minipage}[h]{0.4\linewidth}
                    \frame{\includegraphics[width=1\linewidth]{images/login_code.png}}
                    \caption{После введения Email} %% подпись к рисунку
                    \label{ris:login_code} %% метка рисунка для ссылки на него
                \end{minipage}
                \hfill
                \begin{minipage}[h]{0.4\linewidth}
                    \frame{\includegraphics[width=1\linewidth]{images/login.png}}
                    \caption{Ввод персональных данных}
                    \label{ris:personal_data}
                \end{minipage}
            \end{center}
        \end{figure}

        \item После успешной авторизации пользователь попадает на главный экран.
        Управление происходит через нижнюю панель (Рис. ?), на которой располагаются кнопки:
        \begin{itemize}[noitemsep]
            \item Главная;
            \item Обращения;
            \item Общий чат;
            \item О приложении;
        \end{itemize}
    \end{enumerate}

    \subsubsection{Главная}

    \begin{enumerate}
        \item На главном экране (Рис.~\ref{ris:main_page}) располагаются новости общежития, важные уведомления вверху страницы,
        а также кнопка <<Профиль>> для редактирования профиля.
        \item Список новостей представляет из себя горизонтальную <<карусель>>, новости можно перелистывать движением пальца.
        \item На любую новость можно нажать, чтобы посмотреть более подробную информацию.
        \item Информацию главной страницы можно обновить, потянув страницу вверх до появления индикатора загрузки.

        \item При нажатии на кнопку <<Профиль>> появляется экран (Рис.~\ref{ris:profile_page}) со следующими полями:
        \begin{itemize}[noitemsep]
            \item Имя;
            \item Фамилия;
            \item Общежитие;
            \item Комната.
        \end{itemize}
        В каждое поле, кроме полей <<Имя>> и <<Фамилия>>, можно вносить изменения.
        \begin{figure}[h]
            \begin{center}
                \begin{minipage}[h]{0.33\linewidth}
                    \frame{\includegraphics[width=1\linewidth]{images/main_page.png}}
                    \caption{Главный экран} %% подпись к рисунку
                    \label{ris:main_page} %% метка рисунка для ссылки на него
                \end{minipage}
                \hfill
                \begin{minipage}[h]{0.33\linewidth}
                    \frame{\includegraphics[width=1\linewidth]{images/profile_page.png}}
                    \caption{Профиль}
                    \label{ris:profile_page}
                \end{minipage}
            \end{center}
        \end{figure}

        \item После изменений требуется нажать на кнопку <<Сохранить>>.
        \item После сохранения изменений профиль пользователя попадает в очередь на модерацию введённых данных.

        \includegraphics{example.2}

    \end{enumerate}


    \begin{thebibliography}{3}
        \bibitem{gost}Единая система программной документации – М.: ИПК, Издательство стандартов, 2000, 125 стр.
        \bibitem{lms} LMS [Электронный ресурс] URL: \url{https://lms.hse.ru/} (Дата обращения: 30.04.2020, режим доступа: свободный)
        \bibitem{ruz} РУЗ [Электронный ресурс] URL: \url{https://ruz.hse.ru/} (Дата обращения: 30.04.2020, режим доступа: свободный)
        \bibitem{github} GitHub [Электронный ресурс] URL: \url{https://github.com/} (Дата обращения: 30.04.2020, режим доступа: свободный)
        \bibitem{microsoft} Документация Microsoft [Электронный справочник] URL: \url{https://docs.microsoft.com/ru-ru/} (Дата обращения: 30.04.2020, режим доступа: свободный)
        \bibitem{stackoverflow} StackOverflow [Электронный ресурс] URL: \url{https://stackoverflow.com/} (Дата обращения: 30.04.2020, режим доступа: свободный)
    \end{thebibliography}


    \registrationList

\end{document}
