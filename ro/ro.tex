\documentclass{../includes/TechDoc}
\usepackage[T1]{fontenc}
\usepackage[utf8]{inputenc}

\title{Мобильное приложение для сообщений о бытовых проблемах студентов в общежитии}
\author{Студент группы БПИ-194}{В. А. Анненков}
\academicTeacher{Доцент департамента программной инженерии факультета компьютерных наук}{Х. М. Салех}

\documentTitle{Руководство оператора}
\documentCode{RU.17701729.02.07-01 ТЗ 01--1}

\begin{document}
    \maketitle

    \tableofcontents


    \section{Назначение программы}

    \subsection{Функциональное назначение}

    Функциональным назначением программы является помощь и консультирование студентов по вопросам, связанным
    с проживанием в общежитии. Студент может задать вопрос агенту поддержки, пообщаться в общем чате с другими
    студентами из этого же общежития, узнать последние новости общежития. Агент поддержки имеет доступ к панели
    администрирования, через которую он может создавать объявления, модерировать пользователей и их действия. Отвечать
    на обращения пользователей агент поддержки может с помощью мобильного приложения.

    \subsection{Эксплуатационное назначение}

    Программа может быть использована высшим учебным заведением для обеспечения централизованной связи со студентами, а
    также для информирования студентов о последних новостях в своих общежитиях.

    \subsection{Состав выполняемых функций}

    Программа должна обеспечивать возможность выполнения следующих функций:\\

    Для клиента:
    \begin{itemize}[noitemsep]
        \item регистрация и авторизация;
        \item просмотр страницы с новостями;
        \item просмотр и редактирование собственного профиля;
        \item просмотр/создание/удаление/дополнение своих обращений;
        \item просмотр и взаимодействие с общим чатом.
    \end{itemize}

    Для агента поддержки:
    \begin{itemize}[noitemsep]
        \item регистрация и авторизация в приложении;
        \item регистрация и авторизация в панели администрирования;
        \item просмотр и изменение списка новостей;
        \item просмотр/изменение/удаление/дополнение любых обращений.
    \end{itemize}

    Для администратора:
    \begin{itemize}[noitemsep]
        \item всё, что доступно агенту поддержки;
        \item управление существующими агентами поддержки;
        \item возможность назначить или отозвать у пользователя статус агента поддержки;
        \item полный доступ без ограничений к панели администрирования.
    \end{itemize}


    \section{Условия выполнения программы}

    \subsection{Минимальный состав аппаратных средств}

    Для исправной работы программы требуются следующие характеристики:
    \begin{itemize}[noitemsep]
        \item мобильный телефон с операционной системой iOS;
        \item не менее 9 МБ оперативной памяти;
        \item интернет-соединение.
    \end{itemize}

    \subsection{Минимальный состав программных средств}

    Для работы программы необходим следующий состав программных средств:
    \begin{itemize}[noitemsep]
        \item операционная система iOS 11 и выше.
    \end{itemize}

    \subsection{Требования к пользователю}

    Для работы с данной программой от конечного пользователя требуются базовые навыки владения
    операционной системой iOS.


    \section{Выполнение программы}

    \subsection{Запуск программы}

    \ldots

    \subsection{Описание интерфейса программы}

    \ldots


    \section{Используемая литература}

    \begin{enumerate}
        \item Документация Microsoft, электронный справочник: https://docs.microsoft.com/ru-ru/
        \item StackOverflow, электронный ресурс: https://stackoverflow.com/
        \item Список дополняется\ldots
    \end{enumerate}


    \registrationList

\end{document}
