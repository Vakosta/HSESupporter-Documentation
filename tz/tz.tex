\documentclass{../includes/TechDoc}
\usepackage[T1]{fontenc}
\usepackage[utf8]{inputenc}

\title{Мобильное приложение для сообщений о бытовых проблемах студентов в общежитии}
\author{Студент группы БПИ-194}{В. А. Анненков}
\academicTeacher{Доцент департамента программной инженерии факультета компьютерных наук}{Х. М. Салех}

\documentTitle{Техническое задание}
\documentCode{RU.17701729.02.07-01 ТЗ 01-1}

\begin{document}
    \maketitle

    \tableofcontents

    \section{Введение}

\subsection{Наименование программы}

\subsubsection{Наименование программы на русском языке}

Мобильное приложение для сообщений о бытовых проблемах студентов в общежитии.

\subsubsection{Наименование программы на английском языке}

Mobile Application for Informing about Household Problems in Dormitory.

\subsection{Краткая характеристика области применения}

На момент разработки не существует инструмента, который позволяет пользователю быстро сообщать о проблемах в общежитиях.

Данное приложение нацелено на студентов, живущих в общежитиях, и позволяет быстро сообщать о проблемах бытового и социального характеров.

\subsection{Основание для разработки}

Приказ декана факультета компьютерных наук Национального исследовательского университета «Высшая школа экономики» № ХХХ от ХХ.ХХ.ХХХХ <<Об утверждении тем, руководителей курсовых работ студентов образовательной программы Программная инженерия факультета компьютерных наук>>.


    \section{Назначение разработки}

\subsection{Функциональное назначение}

Функциональным назначением приложения является обеспечение процесса взаимодействия студентов и агентов технической поддержки, путём текстового общения.

Приложение также предоставляет доступ к общему чату, позволяя студентам быть на связи друг с другом.

\subsection{Эксплуатационное назначение}

Данное приложение может быть использовано студентами для сообщения о социальных и бытовых проблемах в общежитии и для поддержания связи с другими студентами общежития.



    \section{Требования к программе}

    \subsection{Требования к функциональным характеристикам}

    Разрабатываемое приложение должно:

\begin{enumerate}
    \item выполнять авторизацию и регистрацию в сервисе;
    \item отображать экран о проверке заявки на регистрацию;
    \item давать пользователю возможность сообщить о проблеме конкретного типа:
    \begin{enumerate}
        \item бытовая проблема;
        \item социальная проблема;
        \item организационная проблема;
        \item вопрос общего характера.
    \end{enumerate}
    \item давать пользователю доступ к общему чату общежития;
    \item отправлять push-уведомления о следующих событиях:
    \begin{enumerate}
        \item получен ответ от агента технической поддержки;
        \item упоминание пользователя в общем чате;
        \item информационные уведомления, содержащие полезную информацию для всех жителей общежития;
        \item изменения, связанные с аккаунтом пользователя:
        \begin{enumerate}
            \item аккаунт заблокирован;
            \item смена общежития.
        \end{enumerate}
    \end{enumerate}
    \item давать пользователю возможность выбора:
    \begin{enumerate}
        \item цветовой темы:
        \begin{enumerate}
            \item темной;
            \item светлой.
        \end{enumerate}
        \item размера шрифта.
    \end{enumerate}
\end{enumerate}


    \subsection{Требования к интерфейсу}

    Приложение должно иметь интерфейс, позволяющий пользователю работать с программой с минимальной предварительной подготовкой.

Интерфейс должен позволять:
\begin{enumerate}
    \item выполнять регистрацию и авторизацию в сервисе;
    \item смотреть и редактировать персональный профиль пользователя;
    \item создавать новые темы для тех. поддержки;
    \item дополнять сущетсвующие темы новой информацией;
    \item отображать ответы агентов поддержки;
    \item взаимодействовать с общим чатом общежития.
\end{enumerate}


    \subsection{Требования к формату входных и выходных данных}

    \subsubsection{Входные данные}

    Формат входных данных -- нажатие на цифровые кнопки или ввод текста в строку набора сообщения с помощью цифровой или механической клавиатуры, присоединенной к устройству с помощью провода или беспроводным способом.

    \subsubsection{Выходные данные}

    Формат выходных данных -- отображение актуальной информации о вопросах пользователя, а также отображение актуальных сообщений в общем чате общежития.

    \subsection{Условия эксплуатации}

    \subsubsection{Климатические условия}

    Климатические условия сопадают с климатическими условиями эксплуатации устройства.

    \subsubsection{Требования к пользователю}

    Пользователь должен обладать базовыми навыками работы с операционной системой iOS или Android.

    \subsection{Требования к составу и параметрам технических средств}

    Для корректной работы приложения необходимо устройство с оперативной памятью более 2GB и портативной памятью более 50MB.

    \subsection{Требования к информационной и программной совместимости}

    На устройстве должна быть установлена операционная система Android 5.0 и выше или iOS 11 и выше.

    \subsection{Требования к составу сетевых средств}

    У устройства должен быть доступ к сети интернет для скачивания и установки данного приложения, а также для создания тем и взаимодействия с общим чатом.

    \subsection{Требования к маркировке и упаковке}

    Приложение должно размещаться в магазинах <<App Store>> и «Play Market» в соответствии с авторскими правами и может быть загружено оттуда потенциальным пользователем.


    \section{Требования к программной документации}

    \subsection{Состав программной документации}

    В состав программной документации должны входить следующие компоненты:
\begin{enumerate}
    \item Техническое задание (ГОСТ 19.201-78)
    \item Программа и методика испытаний (ГОСТ 19.301-78)
    \item Пояснительная записка (ГОСТ 19.404-79)
    \item Руководство оператора (ГОСТ 19.505-79)
    \item Текст программы (ГОСТ 19.401-78)
\end{enumerate}


    \subsection{Специальные требования к программной документации}

    Документы к программе должны быть выполнены в соответствии с ГОСТ 19.106-78 и ГОСТами к каждому виду документа (см. п. 5.1.);

    Пояснительная записка должна быть загружена в систему Антиплагиат через LMS «НИУ ВШЭ». Лист, подтверждающий загрузку пояснительной записки, сдается в учебный офис вместе со всеми материалами не позже, чем за день до защиты курсовой работы.;

    Вся документация также воспроизводится в печатном виде, она должна быть подписана академическим руководителем образовательной программы 09.03.04 «Программная инженерия», руководителем разработки и исполнителем перед сдачей курсовой работы в учебный офис не позже одного дня до защиты;

    Документация и программа также сдается в электронном виде в формате .pdf или .docx. в архиве формата .zip или .rar;

    За один день до защиты комиссии все материалы курсового проекта:
    \begin{itemize}
        \item техническая документация,
        \item программный проект,
        \item исполняемый файл,
        \item отзыв руководителя
    \end{itemize}
    должны быть загружены одним или несколькими архивами в проект дисциплины «Курсовой проект 2017-2018» в личном кабинете в информационной образовательной среде LMS (Learning Management System) НИУ ВШЭ.


    \section{Стадии и этапы разработки}

    \begin{enumerate}
        \item техническое задание:
        \begin{enumerate}
            \item этапы разработки:
            \begin{enumerate}
                \item обоснование необходимости разработки программы;
                \item постановка задачи;
                \item сбор исходных материалов;
                \item выбор и обоснование критериев эффективности и качества разрабатываемой программы;
                \item обоснование необходимости проведения научно-исследовательских работ;
            \end{enumerate}
            \item разработка и утверждение технического задания:
            \begin{enumerate}
                \item определение требований к программе;
                \item определение стадий, этапов и сроков разработки программы и документации на неё;
                \item согласование и утверждение технического задания;
            \end{enumerate}
        \end{enumerate}
        \item технический проект:
        \begin{enumerate}
            \item разработка технического проекта:
            \begin{enumerate}
                \item уточнение структуры входных и выходных данных;
                \item разработка алгоритма решения задачи;
                \item определение формы представления входных и выходных данных;
                \item разработка структуры программы;
                \item окончательное определение конфигурации технических средств.
            \end{enumerate}
            \item утверждение технического проекта:
            \begin{enumerate}
                \item разработка пояснительной записки;
                \item согласование и утверждение технического проекта.
            \end{enumerate}
        \end{enumerate}
        \item рабочий проект:
        \begin{enumerate}
            \item разработка программы:
            \begin{enumerate}
                \item программирование и отладка программы.
            \end{enumerate}
            \item разработка программной документации:
            \begin{enumerate}
                \item разработка программных документов в соответствии с требованиями гост 19.101-77.
            \end{enumerate}
            \item испытания программы:
            \begin{enumerate}
                \item разработка, согласование и утверждение порядка и методики испытаний;
                \item корректировка программы и программной документации по результатам испытаний.
            \end{enumerate}
        \end{enumerate}
    \end{enumerate}


    \section{Порядок контроля и приёмки}

    Проверка программного продукта, в том числе и на соответствие техническому заданию, осуществляется исполнителем вместе с заказчиком согласно «Программе и методике испытаний», а также пункту 5.2.

    Защита выполненного проекта осуществляется комиссии, состоящей из преподавателей департамента программной инженерии, в утверждённые приказом декана ФКН сроки.


    \section{Технико-экономические показатели}

    \subsection{Предполагаемая потребность}

    Данный продукт будет помогать решать возникающие у студентов проблемы связанные с общежитием.

    \subsection{Экономические преимущества разработки по сравнению с отечественными и зарубежными аналогами}

    На момент начала разработки на рынке не было выявлено аналогичных продуктов.

    \registrationList

\end{document}
