\section{Требования к программной документации}

\subsection{Состав программной документации}

В состав программной документации должны входить следующие компоненты:
\begin{enumerate}
    \item Техническое задание (ГОСТ 19.201-78)
    \item Программа и методика испытаний (ГОСТ 19.301-78)
    \item Пояснительная записка (ГОСТ 19.404-79)
    \item Руководство оператора (ГОСТ 19.505-79)
    \item Текст программы (ГОСТ 19.401-78)
\end{enumerate}


\subsection{Специальные требования к программной документации}

Документы к программе должны быть выполнены в соответствии с ГОСТ 19.106-78 и ГОСТами к каждому виду документа (см. п. 5.1.);

Пояснительная записка должна быть загружена в систему Антиплагиат через LMS «НИУ ВШЭ». Лист, подтверждающий загрузку пояснительной записки, сдается в учебный офис вместе со всеми материалами не позже, чем за день до защиты курсовой работы.;

Вся документация также воспроизводится в печатном виде, она должна быть подписана академическим руководителем образовательной программы 09.03.04 «Программная инженерия», руководителем разработки и исполнителем перед сдачей курсовой работы в учебный офис не позже одного дня до защиты;

Документация и программа также сдается в электронном виде в формате .pdf или .docx. в архиве формата .zip или .rar;

За один день до защиты комиссии все материалы курсового проекта:
\begin{itemize}
    \item техническая документация,
    \item программный проект,
    \item исполняемый файл,
    \item отзыв руководителя
\end{itemize}
должны быть загружены одним или несколькими архивами в проект дисциплины «Курсовой проект 2017-2018» в личном кабинете в информационной образовательной среде LMS (Learning Management System) НИУ ВШЭ.